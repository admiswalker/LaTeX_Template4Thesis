% これを入れることでページ番号が表示されない。
\thispagestyle{empty}

% abstract 環境は jsbook では「概要」と表示してくれないため、手動で表示させる。
% 参照 http://oku.edu.mie-u.ac.jp/tex/mod/forum/discuss.php?d=2121
\begin{center}
  {\large \sf 概要}
\end{center}

ここには論文の概要(abstract)を書きます。論文の先頭なので早い時期に書き始める人がいますが、論文の結論や論理展開はなかなか執筆終盤まで固まりません。そのため、論文の流れや結論がかなり明確になった最終段階で書くようにしましょう。

概要は論文全体の内容を短文で説明するものですので、研究の背景と目的、研究内容、結果と結論などが全て網羅されている必要があります。ここを読んだだけで、論文の中身が大雑把に把握できるようにすることが大切です。原則として改行せずに1段落で書きますが、これは複数段落に分けて書くような文章を無理やり1段落に合体させろということではありません。1段落で流れるように書いてください。


