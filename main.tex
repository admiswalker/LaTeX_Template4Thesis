\documentclass[oneside,openright,uplatex]{jsbook} % oneside: 章ごとに改ページ.twoside: 章の初めが右側となるように改ページ.

% jsbookで余白が広すぎるのを直す
% 参照 https://oku.edu.mie-u.ac.jp/~okumura/jsclasses/
\setlength{\textwidth}{\fullwidth}
\setlength{\evensidemargin}{\oddsidemargin}

% OTF フォントを使えるようにし、複数のウェイトも使用可能にする。
% これがないと、Mac のヒラギノ環境で使われる角ゴが太すぎてみっともない。
\usepackage[deluxe]{otf}

% OT1→T1に変更し、ウムラウトなどを PDF 出力で合成文字ではなくす
\usepackage[T1]{fontenc}

% uplatex の場合に必要な処理 
\usepackage[utf8]{inputenc} % エンコーディングが UTF8 であることを明示する。
\usepackage[prefernoncjk]{pxcjkcat} % アクセントつきラテン文字を欧文扱いにする

% Helvetica と Times を sf と rm のそれぞれで使う。
% default だとバランスが悪いので、日本語に合わせて文字の大きさを調整する。
\usepackage[scaled=1.05,helvratio=0.95]{newtxtext}

\usepackage[dvipdfmx]{graphicx} % 画像の添付に必要
\usepackage[dvipdfmx]{color}    % 画像の添付に必要

% 表でセルを複数列で結合する
\usepackage{multicol}

% 数式の機能を拡張
\usepackage{amsmath}

% citep や citet を有効にする
\usepackage{natbib} % 参考文献を bib で管理するために必要.
\usepackage{url}    % 参考文献のリンク指定オプションに必要.

% (Okumura, 2009) などを (Okumura 2009) とする
\setcitestyle{aysep={}}

% subfigure 環境で、(a)、(b) などの番号を左上に表示する。宇宙系の分野ではこれが一般的なはず。
\usepackage[nooneline]{subfigure}
\subfiguretopcaptrue

% 行番号を表示する。添削時のみに使い、事務提出版ではコメントアウトする
%\usepackage{lineno}
%\linenumbers

% PDF 内で外部リンクや文書内リンクを生成したい場合に使う(好みによる)
% \usepackage[dvipdfmx]{hyperref}

% newcommand を使うことで、繰り返し使う長ったらしい入力を簡単にすることができる
\makeatletter
\newcommand{\ion}[2]{#1$\;${\small\rmfamily\@Roman{#2}}\relax}%
\makeatother
\newcommand{\HI}{\mbox{\ion{H}{1}}} % 中性原子ガス(HI 領域)の例
\newcommand{\bs}{\symbol{92}} % backslash


%%%%%%%%%%%%%%%%%%%%%%%%%%%%%%%%%%%%%%%%%%%%%%%%%%%%%%%%%%%%%%%%%%%%%%%%%%%%%%%%%%%%%%%%%%%%%%%%%%%%%%%%%%%%%%%%%%%%%%%%%%%%%%

% 図を通し番号にする: http://rexpit.blog29.fc2.com/blog-entry-99.html
\usepackage{remreset}	% removefromreset に使う
\makeatletter
	\@removefromreset{figure}{chapter}
	\def\thefigure{\arabic{figure}}

	\@removefromreset{table}{chapter}
	\def\thetable{\arabic{table}}

	\@removefromreset{equation}{chapter}
	\def\theequation{\arabic{equation}}
\makeatother

\usepackage{comment} % \begin{comment} \end{comment}
\usepackage{color}   % {\color{red}メモ}

\usepackage{natbib} % citep や citet を有効にする
\bibpunct[:]{[}{]}{,}{a}{}{,} % 括弧を () -> [] へ変更
%\usepackage[numbers]{natbib} % citep や citet を有効にする
%\bibpunct[:]{(}{)}{,}{a}{}{,}

\usepackage[]{multicol}         % 段組み % \begin{multicols}{2}%2段組み開始 ~ \end{multicols}%2段組み終了

\usepackage{wrapfig}            % 図の回り込みを許す.\begin{figure}~\end{figure} -> \begin{wrapfigure}{r}{90mm}~\end{wrapfigure} へ置き換え
                                % 直後の文が制御記号だと,回り込みに失敗し,文字が図下に重なるので,制御記号の直前に ''\noindent \ \ '' を挿入して回避する.

%%%%%%%%%%%%%%%%%%%%%%%%%%%%%%%%%%%%%%%%%%%%%%%%%%%%%%%%%%%%%%%%%%%%%%%%%%%%%%%%%%%%%%%%%%%%%%%%%%%%%%%%%%%%%%%%%%%%%%%%%%%%%%

\begin{document}

\begin{titlepage}


\vspace*{120truept}
\begin{center}
  \huge{学 \ 士 \ 論 \ 文 / \ 修 \ 士 \ 論 \ 文}\\
  \vspace{30truept}
  \huge{タイトルが\\長い場合はいい感じに改行}\\ % title
\vspace{100truept}
\LARGE{xx大学大学院xxxx研究科}\\
\LARGE{xxxx専攻xxxx分野xx研究室}\\
\LARGE{苗字 名前}\\
\end{center}

%\newpage
%\thispagestyle{empty} % このページのページ番号を表示しない.
%\ % add emptypage

\end{titlepage}
 % 表紙
\thispagestyle{empty} % ページ番号を表示しない.

% レイアウトを他の章のはじめのページと揃える.
 \\
\\
\\
\\
\noindent{\Huge \sf Abstract/要旨}\\
\\
\\
\\

Abstractを書きます.




\frontmatter     % make page number roman
\tableofcontents % 目次
%\listoffigures  % 図目次
%\listoftables   % 表目次

\mainmatter      % make page number arabic
\chapter{Introduction/序論}
\label{chap_Introduction}

\section{Introduction}
イントロダクション.

\section{Review}
先行研究.

\section{Purpose}
研究目的.


\chapter{Device}
\label{chap_Device}

実験装置/観測装置について説明する.タイトルは実験/観測装置の名称などにする.



\chapter{Method}
\label{chap_Method}

実験手法/解析手法/等について説明する.



\chapter{Implementation}
\label{chap_Implementation}

Implementation/実装について説明する.



\chapter{Results/Benchmarks/結果}
\label{chap_Results}

Results/Benchmarksについて記述する.



\chapter{Discussion/考察}
\label{chap_Discussion}

Discussion/考察について記述する.



\chapter{Conclusion/結論}
\label{chap_Conclusion}

Conclusion/結論について記述する.



\chapter*{Appendix/付録} % 章番号を出さない
\addcontentsline{toc}{chapter}{Appendix/付録} % 目次に載せる

Appendix/付録.

% 付録は chapter の 1 つとして作りますが、章番号は表示しません。
% また付録の 1 つずつはアルファベットで番号付けをするのが一般的です。
\setcounter{section}{0} % section の番号をゼロにリセットする
\renewcommand{\thesection}{\Alph{section}} % 数字ではなくアルファベットで数える
\setcounter{equation}{0} % 式番号を A.1 のようにする
\renewcommand{\theequation}{\Alph{section}.\arabic{equation}}
\setcounter{figure}{0} % 図番号
\renewcommand{\thefigure}{\Alph{section}.\arabic{figure}}
\setcounter{table}{0} % 表番号
\renewcommand{\thetable}{\Alph{section}.\arabic{table}}

\section{セクション1}
内容.

\section{セクション2}
図/表など.

\chapter*{Acknowledgments/謝辞}%
\addcontentsline{toc}{chapter}{Acknowledgments/謝辞}

感謝の気持ちを述べる.




\renewcommand{\bibname}{参考文献} % jecon.bst だと bib の doi を読み込めずエラーを吐くので,ここだけ,削除すること.
\bibliographystyle{jecon}
\nocite{*} % 文献の引用がない場合のエラーを抑制.
\bibliography{10_References}

\end{document}
